\documentclass[referat]{SCWorks}

\usepackage{preamble}

\begin{document}

% Кафедра (в родительном падеже)
\chair{математической кибернетики и компьютерных наук}

% Тема работы
\title{Компьютерная графика}

% Курс
\course{1}

% Группа
\group{111}

% Специальность/направление код - наименование
\napravlenie{02.03.02 "--- Фундаментальная информатика и информационные технологии}

\department{факультета Компьютерных наук и информационных технологий (КНиИТ)}

\studenttitle{студентов}

% Фамилия, имя, отчество в родительном падеже
\author{Алексеева Никиты Игоревича \\ Зинченко Елены Валентиновны \\ Неклюдовой Марины Сергеевны}

% Научный руководитель (для реферата преподаватель проверяющий работу)
\satitle{доцент, к.\,пед.\,н.} % должность, степень, звание
\saname{А.\,П.\,Грецова}

% Год выполнения отчета
\date{2025}

\maketitle

% Включение нумерации рисунков, формул и таблиц по разделам (по умолчанию -
% нумерация сквозная) (допускается оба вида нумерации)
\secNumbering

\tableofcontents

\intro

% текст

\section{Основы компьютерной графики}

\subsection{Введение в компьютерную графику}

\input{src/section1/introductionGrafics.tex}

\subsection{Типы компьютерной графики}

% текст

\subsection{Математические основы}

% текст

\subsection{Этапы рендеринга}

Рендеринг компьютерной графики включает несколько этапов: 
\begin{enumerate}
    \item Моделирование. Создание 3D"=модели объекта или сцены. На этом этапе создают геометрию объектов с помощью полигонов, кривых и других элементов. Используют специализированное программное обеспечение.  
    \item Текстурирование. Добавление текстур и материалов к модели. Текстуры накладывают на модели, чтобы добавить цвет, узоры и детали поверхности. Для точного наложения текстур 3D"=модель разворачивают на плоскую поверхность "--- это называется UV"=развёрткой.
    \item Освещение. Настройка источников света в сцене.  Освещение определяет, как свет взаимодействует с объектами и материалами. Используют различные типы источников света: точечные, направленные, окружающие. Также рассчитывают тени, которые создают объекты при взаимодействии с источниками света.  
    \item Растеризация или трассировка лучей. Преобразование 3D"=сцены в 2D"=изображение с помощью выбранного метода. В зависимости от выбранного метода, происходит растеризация или трассировка лучей.  
    
    \textbf{Растеризация} "--- это одна из групп методов рендеринга, то есть превращение форм трёхмерной сцены в проекцию на двухмерной сетке пикселей, которая выводится на экран пользователя. 
    
    \textbf{Трассировка лучей} "--- более сложный, но реалистичный подход, который отслеживает путь лучей света от камеры до объектов в сцене, учитывая отражения, преломления и тени.
    \item Постобработка. Финальная обработка изображения для улучшения визуального качества.  На этапе постобработки применяют дополнительные эффекты, такие как размытие, наложение фильтров, коррекция цвета. Также могут добавлять эффекты глубины резкости и смягчения, чтобы повысить визуальное качество изображения.
\end{enumerate}

Рассомтрим основы освещения на примере некоторых моделей.

\textbf{Модель Ламберта} (Рисунок \ref{fig:lambert}) моделирует идеальное диффузное освещение. Считается, что свет при попадании на поверхность рассеивается равномерно во все стороны. При расчете такого освещения учитывается только ориентация поверхности (нормаль $N$) и направление на источник света (вектор $L$). Рассеянная составляющая рассчитывается по закону косинусов. Для удобства все векторы, описанные ниже, берутся единичными. В этом случае косинус угла между ними совпадает со скалярным произведением:

\begin{equation}
    I_{d} = k_{d}\cos(\vec{L}, \vec{N})i_{d} = k_{d}(\vec{L}, \vec{N})i_{d}
\end{equation}
где $I_{d}$ "--- рассеянная составляющая освещенности в точке, $k_{d}$ "--- свойство материала воспринимать рассеянное освещение, $i_{d}$ "--- мощность рассеянного освещения, $\vec{L}$ "--- направление из точки на источник, $\vec{N}$ "--- вектор нормали в точке\cite{light_models}.

Модель Ламберта является одной из самых простых моделей освещения. Она очень часто используется в комбинации других моделей, практически в любой другой модели освещения можно выделить диффузную составляющую. Более"=менее равномерная часть освещения (без присутствия какого"=либо всплеска) как правило будет представляться моделью Ламберта с определенными характеристиками. Данная модель может быть очень удобна для анализа свойств других моделей (за счет того, что ее легко выделить из любой модели и анализировать оставшиеся составляющие).

\textbf{Модель Фонга} (Рисунок \ref{fig:fong}) "--- классическая модель освещения. Она представляет собой комбинацию диффузной составляющей (модели Ламберта) и зеркальной составляющей и работает таким образом, что кроме равномерного освещения на материале может еще появляться блик. Местонахождение блика на объекте, освещенном по модели Фонга, определяется из закона равенства углов падения и отражения. Если наблюдатель находится вблизи углов отражения, яркость соответствующей точки повышается.

Падающий и отраженный лучи лежат в одной плоскости с нормалью к отражающей поверхности в точке падения, и эта нормаль делит угол между лучами на две равные части. То есть отраженная составляющая освещенности в точке зависит от того, насколько близки направления на наблюдателя и отраженного луча. Это можно выразить следующей формулой:

\begin{equation}
    I_{s} = k_{s}\cos^{\alpha}(\vec{R}, \vec{V})i_{s} = k_{s}(\vec{R}, \vec{V})^{\alpha}i_{s}
\end{equation}
где $I_{s}$ "--- зеркальная составляющая освещенности в точке, $k_{s}$ "--- коэффициент зеркального отражения, $i_{s}$ "--- мощность зеркального освещения, $\vec{R}$ "--- направление отраженного луча, $\vec{V}$ "--- направление на наблюдателя, $\alpha$ "--- коэффициент блеска, свойство материала\cite{light_models}.

\section{Инструменты и библиотеки C++ для графики}
\subsection{Обзор популярных библиотек}
\textbf{Графические библиотеки} "--- это наборы заранее написанного кода, которые упрощают процесс отрисовки графики в приложениях. Они предоставляют набор функций для выполнения распространённых графических задач, что значительно сокращает объём шаблонного кода, который приходится писать разработчикам. Также они позволяют программистам сосредоточиться на разработке своих приложений, а не на низкоуровневых деталях рендеринга графики.

Существует множество разных графических библиотек, каждая из которых имеет свои плюсы и минусы, а также подходит для определённых задач. Поэтому для решения вашей задачи для начала нужно определиться с библиотекой, зная её преимущества и недостатки. Перечислим примеры таких библиотек для C++ и остановимся на некоторых из них, чтобы рассмотреть их подробнее \cite{graphics_libraries}:
\begin{itemize}
    \item SDL;
    \item SFML;
    \item OpenGL;
    \item Allegro;
    \item Juce;
    \item Qt и другие.
\end{itemize}

\textbf{SFML} (Simple and Fast Multimedia Library) "--- это одна из самых быстрых и простых библиотек C++ для реализации 2D"=графики, которая является мультимедийной и позволяет работать с системой, непосредственно с графикой, окнами, аудио и сетью, что помогает упростить разработку игр и мультимедийных приложений. Кроме того она является мультиплатформенной и может быть запущена на распространённых операционных системах: Windows, Linux, macOS, Android и iOS. SFML представленна и на других языках кроме C++: C, .NET, Java, Ruby, Python, Go и других, что позволяет расширить использование данной библиотеки в разных проектах. \cite{sfml_doc}

Стоит выделить главные преимущества SFML:
\begin{itemize}
    \item объектно-ориентированный дизайн, упрощающий использование и понимание;
    \item встроенная поддержка различных графических элементов, таких как фигуры, спрайты и текст;
    \item отлично подходит для начинающих и небольших и средних проектов.
\end{itemize}

\textbf{SDL} (Simple DirectMedia Layer) "--- это кроссплатформенная библиотека для разработки, предназначенная для низкоуровневого доступа к аудио, элементам ввода и графическому оборудованию. SDL поддерживается на всех доступных платформах (Windows, macOS, Linux, iOS и Android), а также возможно использование на разных языках: C\#, Python, Rust и другие. Используется библиотека для разработки игр, видеоплееров и других мультимедийных приложений. \cite{sdl_doc}

Некоторые возможности SDL:
\begin{itemize}
    \item аппаратно-ускоренный вывод 2D- и 3D-графики;
    \item простота интеграции с другими библиотеками, такими как OpenGL, \\ Direct3D или Vulkan;
    \item поддержка шейдеров, текстур и других средств работы с графикой. 
\end{itemize}

\textbf{OpenGL} (Open Graphics Library) "--- это библиотека, которая является одной из самых популярных графических \textbf{прикладных программных интерфейсов} (API – Application Programming Interface) для разработки приложений в области 2D"= и 3D"=графики. Она была разработана ещё в 1992 году на основе библиотеки IRIS GL и до 2017 года активно поддерживалась и развивалась до появления Vulkan API. Однако OpenGL всё ещё широко применяется в различных сферах. Стоит отметить, что библиотека сложнее SFML и SDL, но предлагает широкие возможности и гибкость для продвинутых графических приложений. \cite{opengl_doc}

Рассмотрим особеннсоти OpenGL.
\begin{itemize}
    \item \textit{Стабильность.} Дополнения и изменения реализуются таким образом, чтобы сохранить совместимость с разработанным ранее программным обеспечением.
    \item \textit{Мультиплатформенность} Приложения, использующие OpenGL, гарантируют одинаковый визуальный результат вне зависимости от типа используемой операционной системы и организации отображения информации. Кроме того, эти приложения могут выполняться как на персональных компьютерах, так и на рабочих станциях и суперкомпьютерах.
    \item \textit{Легкость применения.} Стандарт OpenGL имеет продуманную структуру и интуитивно понятный интерфейс, что позволяет с меньшими затратами создавать эффективные приложения, содержащие меньше строк кода, чем с использованием других графических библиотек.
\end{itemize}
\subsection{Практические примеры на SFML}
Познакомимся с библиотекой SFML на простых практических примерах. Для этого нужно скачать бибилотеку на компьютер с официального сайта: \url{https://www.sfml-dev.org/download/}, затем понадобиться провести подключение всех компонентов библиотеки в рабочую область.

Рассмотрим простейшие функции, которые помогут написать первую тестовую программу (вывод простых фигур на экран). Для начала понадобится подключить заголовочный файл, позволяющий работать непосредственно с графикой:

\begin{lstlisting}[style=myStyle, numbers=none]
#include <SFML/Graphics.hpp>
\end{lstlisting}

После подключения появляется возможность использовать новые функции. В \texttt{main} инициализируем новое окно размером 800 на 600 пикселей следующим образом:

\begin{lstlisting}[style=myStyle, numbers=none]
sf::RenderWindow window(sf::VideoMode(800, 600), "Shapes");
\end{lstlisting}

Теперь определимся с формой, размером, цветом и позицией отображаемых фигур и инициализируем их, задав нужные параметры с помощью простых и интуитивно понятных методов:

\begin{lstlisting}[style=myStyle]
sf::CircleShape circle(50.f);
circle.setFillColor(sf::Color::Red);
circle.setPosition(100.f, 100.f);

sf::RectangleShape rect(sf::Vector2f(120.f, 60.f));
rect.setFillColor(sf::Color::Blue);
rect.setPosition(300.f, 200.f);
\end{lstlisting}

После инициализации всех необходимых элементов можно приступить к созданию \textbf{основного цикла}, который реализует всю логику программы и обеспечивает её работу. Он включает в себя такие этапы как: обработка событий, обновление логики, отрисовка. В нашей программе отсутствует логика, поэтому мы пропускаем второй этап:

\begin{lstlisting}[style=myStyle]
while (window.isOpen()) {
    sf::Event event;
    while (window.pollEvent(event)) {
        if (event.type == sf::Event::Closed)
            window.close();
    }

    window.clear(sf::Color::Black);
    window.draw(circle);
    window.draw(rect);
    window.display();
}
\end{lstlisting}

Обязательным элементом является наличие цикла, обрабатывающего события "--- \texttt{\color{blue}while} \texttt{window.pollEvent(event)}, если бы его не было, то программу было бы невозможно просто так закрыть. Полный код (Листинг \ref{lst:shapes}) и результат программы (Рисунок \ref{fig:shapes}).


\section{Реализация графических алгоритмов}

\subsection{Алгоритмы растеризации}

% текст

\subsection{Заливка и работа с цветом}

% текст

\subsection{Оптимизация графики}

\textbf{FPS} (Frames Per Second) "--- это количество кадров в секунду, которое отображается на экране во время воспроизведения видео или игры. Показатель FPS отражает, насколько плавно воспринимается изображение глазом пользователя. Чем выше значение FPS, тем более плавным и комфортным выглядит движение на экране.

Особенно важную роль FPS играет в играх. Так, FPS в играх "--- это показатель, который указывает, сколько кадров рендерится и отображается на экране за одну секунду. Благодаря высокому показателю FPS, картинка в играх является более плавной и приятной глазу. Однако, всё не так линейно, и здесь есть нюансы. Так, важен не только сам показатель в игре, но и его стабильное поддерживание. На устройстве или в игре в процессе может возникнуть проблема, из"=за которой FPS может резко снизиться. Если ошибки происходят периодически, то это гораздо хуже сказывается на качестве игры, чем низкий, но стабильный FPS.

Немаловажную роль в поддержке FPS играют показатели самой техники, то есть мониторов, с которых происходит воспроизведение. Частота обновления экрана оказывает сильное влияние на то, как быстро будет обновляться картинка. Например, частота экрана в 60 Гц способна менять кадр каждый 16 миллисекунд, а в 144 Гц "--- каждые 6 миллисекунд.

Разумеется, такое понятие, как FPS применяется и в съемке кино. Самая распространённая частота кадров в фильмах "--- это 24 FPS. Именно данный показатель является самым комфортным для зрителя при просмотре фильма. Показатель FPS в 24 единицы поддерживает реалистичность и плавность повествования, так как именно с такой частотой видят наши глаза в повседневной жизни. Усиленная плавность может казаться \flqq фальшивой\frqq\, так как в такой ситуации в сцене мы замечаем гораздо больше элементов, чем могли бы сделать это в реальности. Также у такой низкой частоты есть и практическая цель: съемка и монтаж видео с более высокой частотой кадров требуют больше места и производительности.

Если в случае с играми, плавность играет очень важную роль для поддержания качества игры и мгновенной отдачи действия игрока, то главная задача FPS в фильмах "--- это поддержание реалистичности кадра и сохранения художественного стиля.

\textbf{Двойная буферизация} "--- это метод рендеринга, при котором используются два буфера (то есть области памяти) для отображения графики. \textbf{Передний буфер} (front buffer) "--- отображается на экране, \textbf{задний буфер} (back buffer) "--- используется для рендеринга следующего кадра. После завершения рендеринга кадра в заднем буфере происходит обмен: задний буфер становится передним, передний буфер, в свою очередь, становится задним для следующего цикла рендеринга.

Преимущества двойной буферизации:
\begin{itemize}
    \item Плавность анимации. Кадры отображаются целиком, что очень важно для игр и интерактивных приложений.
    \item Без двойной буферизации обновление экрана может происходить во время рендеринга, из-за чего может быть частичное отображение кадров.
\end{itemize}

Недостатки:
\begin{itemize}
    \item Увеличение использования памяти. Используется в 2 раза больше видеопамяти.
    \item Задержки. Если рендеринг кадра занимает больше времени, чем период обновления монитора, частота кадров падает.
\end{itemize}

\textbf{Вертикальная синхронизация} (V"=Sync) "--- это технология, предназначенная для синхронизации частоты обновления монитора с частотой кадров, которую генерирует видеокарта. Основная цель V"=Sync "--- устранить разрывы изображения, которые могут возникать, когда видеокарта выводит кадры быстрее, чем монитор может их отображать.

Когда видеокарта генерирует кадры быстрее, чем монитор может их отображать, происходит разрыв изображения. Это означает, что на экране одновременно отображаются части нескольких кадров, что приводит к визуальным артефактам. V"=Sync решает эту проблему, заставляя видеокарту ждать, пока монитор не будет готов к отображению следующего кадра. Это достигается путем синхронизации частоты обновления монитора с частотой кадров видеокарты. Например, если монитор обновляется с частотой 60 Гц, V"=Sync заставляет видеокарту генерировать кадры с частотой не более 60 FPS, что предотвращает разрывы изображения.

Преимущества вертикальной синхронизации:
\begin{itemize}
    \item Устранение разрывов изображения. Основное преимущество "--- это устранение разрывов изображения, что делает игровой процесс более плавным и приятным для глаз.
    \item Стабильность изображения. Помогает поддерживать стабильную частоту кадров, что может быть полезно в играх, где важна плавность и стабильность.
\end{itemize}

Недостатки:
\begin{itemize}
    \item Задержка ввода (input lag). Одним из основных недостатков является увеличение задержки ввода. Это может быть критично в играх, где важна быстрая реакция, таких как шутеры или файтинги. Задержка ввода может привести к тому, что действия игрока будут отображаться на экране с небольшим запозданием, что может повлиять на игровой процесс.
    \item Падение производительности. В некоторых случаях может приводить к падению производительности, особенно если видеокарта не может поддерживать стабильную частоту кадров, равную частоте обновления монитора. В таких ситуациях частота кадров может резко падать.
\end{itemize}

\conclusion

% текст

% Отобразить все источники. Даже те, на которые нет ссылок.
% \nocite{*}

\bibliographystyle{ugost2003}
\bibliography{thesis}

% Окончание основного документа и начало приложений Каждая последующая секция
% документа будет являться приложением
\appendix

\section{}

\begin{figure}[H]
    \centering
    \includegraphics[width=0.8\textwidth]{src/img/lambert.jpg}
    \caption{Модель Ламберта}
    \label{fig:lambert}
\end{figure}

\begin{figure}[H]
    \centering
    \includegraphics[width=0.8\textwidth]{src/img/fong.jpg}
    \caption{Модель Фонга}
    \label{fig:fong}
\end{figure}
\newpage

\section{}

\begin{figure}[ht]
    \centering
    \includegraphics[width=0.8\textwidth]{src/img/sfml_shapes.png}
    \caption{Shapes}
    \label{fig:shapes}
\end{figure}

\lstinputlisting[style=myStyle, frame=single, rulecolor=\color{blue!30}, caption=Shapes, captionpos=b, label=lst:shapes]{src/codes/1.txt}

\begin{figure}[ht]
    \centering
    \includegraphics[width=0.8\textwidth]{src/img/sfml_events.png}
    \caption{Events}
    \label{fig:events}
\end{figure}

\lstinputlisting[style=myStyle, frame=single, rulecolor=\color{blue!30}, caption=Events, captionpos=b, label=lst:events]{src/codes/2.txt}

\begin{figure}[ht]
    \centering
    \includegraphics[width=0.8\textwidth]{src/img/sfml_animation.png}
    \caption{Animation}
    \label{fig:animation}
\end{figure}

\lstinputlisting[style=myStyle, frame=single, rulecolor=\color{blue!30}, caption=Animation, captionpos=b, label=lst:animation]{src/codes/3.txt}

\end{document}
