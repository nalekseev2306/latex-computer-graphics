\documentclass[referat]{SCWorks}

\usepackage{preamble}

\begin{document}

% Кафедра (в родительном падеже)
\chair{математической кибернетики и компьютерных наук}

% Тема работы
\title{Компьютерная графика}

% Курс
\course{1}

% Группа
\group{111}

% Специальность/направление код - наименование
\napravlenie{02.03.02 "--- Фундаментальная информатика и информационные технологии}

\studenttitle{студентов}

% Фамилия, имя, отчество в родительном падеже
\author{Алексеева Никиты Игоревича \\ Зинченко Елены Валентиновны \\ Неклюдовой Марины Сергеевны}

% Научный руководитель (для реферата преподаватель проверяющий работу)
\satitle{доцент, к.\,пед.\,н.} %должность, степень, звание
\saname{А.\,П.\,Грецова}

% Год выполнения отчета
\date{2025}

\maketitle

% Включение нумерации рисунков, формул и таблиц по разделам (по умолчанию -
% нумерация сквозная) (допускается оба вида нумерации)
\secNumbering

\tableofcontents

\intro

% текст

\section{Основы компьютерной графики}

\subsection{Введение в компьютерную графику}

% текст

\subsection{Типы компьютерной графики}

% текст

\subsection{Математические основы}

% текст

\subsection{Этапы рендеринга}

% текст

\section{Инструменты и библиотеки C++ для графики}

\subsection{Обзор популярных библиотек}

% текст

\subsection{Настройка среды разработки}

% текст

\subsection{Практические примеры на SFML}

% текст

\subsection{Переход к 3D: основы OpenGL}

% текст

\section{Реализация графических алгоритмов}

\subsection{Алгоритмы растеризации}

% текст

\subsection{Заливка и работа с цветом}

% текст

\subsection{Оптимизация графики}

% текст

\conclusion

% текст

% Отобразить все источники. Даже те, на которые нет ссылок.
% \nocite{*}

\bibliographystyle{ugost2003}
\bibliography{thesis}

% Окончание основного документа и начало приложений Каждая последующая секция
% документа будет являться приложением
\appendix

\end{document}
