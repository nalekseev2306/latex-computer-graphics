C++ играет важную роль в компьютерной графике благодаря производительности, низкоуровневому управлению ресурсами и интеграции с графическими библиотеками, такими как: SFML, OpenGL, SDL. На примерах реализации алгоритмов Брезенхэма, матричных преобразований и оптимизационных методов (двойная буферизация, V-Sync) показано, как язык обеспечивает контроль над рендерингом и аппаратными возможностями.

Использование C++ позволяет напрямую применять математические модели (цветовые пространства, координатные системы) в коде, что используется в задачах визуализации. На примере библиотеки SFML можно сделать вывод о стабильности FPS за счёт эффективного управления памятью и событиями.

Перспективные направления, например трассировка лучей, VR/AR, требуют ещё большей оптимизации, где C++ незаменим благодаря поддержке параллельных вычислений и низкоуровневых API.