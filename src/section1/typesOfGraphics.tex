\textbf{Растровая графика}. Работает с изображениями. Изображение разбивается на точки "--- \textbf{пиксели}. \textbf{Пиксель} "--- самая маленькая единица цифрового изображения. Каждый пиксель имеет свой цвет и яркость в изображении. Пиксели образуют строки и столбцы, имеют определённые координаты (номер строки и столбца как ячейки таблицы), образуя растровое изображение. Преимущество растровой графики "--- это высокая точность передачи цветов и полутонов. Качество изображения напрямую зависит от количества пикселей в нём. Чем больше пикселей, тем больше деталей можно отобразить. Изображение в растровой графике формируется в процессе сканирования многоцветных иллюстраций и фотографий или при использовании цифровых фото"= видеокамер. Кроме того, растровое изображение можно создать на компьютере, используя растровые графические редакторы. \cite{burceva2023cg}

Недостатки растровой графики:
\begin{itemize}
    \item Большой объём изображения, так как необходимо хранить код цвета каждого пикселя;
    \item Растровые изображения чувствительны к масштабированию. При уменьшении изображения несколько соседних пикселей преобразуются в один, вследствие чего теряется чёткость маленьких деталей изображения. При увеличении изображения, наоборот, пиксели добавляются, в результате чего, несколько соседних имеют одинаковый цвет, появляется ступенчатый эффект.
\end{itemize}

\textbf{Векторная графика}. Изображение состоит из ряда элементарных геометрических объектов, называемых \textbf{примитивами} "--- точек, линий, кривых Без\\ье, кругов, окружностей, многоугольников. Каждый объект задаётся координатами опорных точек и математическими формулами рисования объекта. Каждому объекту также можно указать цвет, толщину, стиль линии контура (сплошная, пунктирная и т.д.). В итоге, можно хранить не все точки изображения, а координаты узлов объектов и их свойства (цвет, связь с другими узлами и т.д.). Векторное изображение можно перевести в растровое (растеризация), в обратную сторону это не работает, или требует значительных вычислительных мощностей и времени. \cite{burceva2023cg}

Недостатки векторной графики:
\begin{itemize}
    \item Получить фотореалистичное изображение трудно, векторные изображения выглядят искусственно;
    \item Некоторые сложные объекты сложно создать с помощью векторной графики, так как необходимо использовать большое количество объектов, что значительно повышает объём памяти, отведённый на изображение, и время на отображение.
\end{itemize}

Преимущества:
\begin{itemize}
    \item Масштабирование рисунка без потерь качества (так как происходит пересчёт сравнительно небольшого числа координат узлов);
    \item Объём файлов существенно меньше, чем у растрового изображения.
\end{itemize}

Рассмотрим математическое описание объектов векторной графики. В векторной графике линия "--- основной элемент, а объекты формируются из сегментов, то есть линий между узлами: \cite{vec_graphics}
\begin{enumerate}
    \item Точка (узел) задаётся координатами ($x$, $y$).
    \item Прямая описывается, известным многим, уравнением $y = ax + b$ (с двумя параметрами: $a$, $b$). Отрезок прямой требует дополнительно координат начала и конца отрезка соответственно.
    \item Кривые второго порядка (параболы, эллипсы, окружности) описываются уравнениями уже с 5 параметрами: $x^{2} + a_{1}y^{2} + a_{2}xy + a_{3}x + a_{4}y + a_{5} = 0$. Для отрезка кривой нужно 2 дополнительных параметра.
    Прямые и кривые второго порядка "--- это частные случаи кривых третьего порядка.
    \item Кривые третьего порядка (например, $y = x^{3}$) могут иметь точки перегиба, а поэтому требуют 9 параметров.
    \item Кривые Безье "--- это также частный случай кривых третьего порядка. Для их описания нужно 8 параметров. Они управляются касательными (виртуальными рычагами), которые задают форму кривой. Касательные можно перемещать, меняя угол и длину, что позволяет гибко моделировать сложные формы. Кривые Безье упрощают создание плавных линий.
\end{enumerate}

\textbf{3D"= или трёхмерная графика}. Создание трёхмерного объекта помогает достигать его реалистичности. 3D"=объекты широко применяются в веб"=дизайне, интерфейсах мобильных приложений, виртуальной и дополненой реальности, в компьютерных играх, рекламе и т.д. Создание объекта происходит в 3 этапа.
\begin{enumerate}
    \item Построение модели создаваемого объекта (создание его формы).
    \item Размещение объекта в пространстве (создание макета и анимации).
    \item Доработка объекта до желаемого вида.
\end{enumerate}

Объекты чаще всего создаются с помощью полигонов (поверхности, которые задаются точками). То есть сначала модель объекта представлена сеточным каркасом из точек. \textbf{Рендеринг в 3D"=графике} "--- это процесс преобразования 3D"=моделей, сцен и данных (геометрия, текстуры, освещение) в 2D"=изображение, которое можно отобразить на экране. Имеет место в создании визуала для игр, фильмов и т.д.

Недостатки 3D"=графики:
\begin{itemize}
    \item Большой объём файлов;
    \item Большие временные затраты на создание модели объекта.
\end{itemize}