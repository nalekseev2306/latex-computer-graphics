\textbf{Декартова система координат} (ДСК). Для получения изображения на экране компьютера, необходимо использовать какой"=либо способ математического описания объектов в трёхмерном пространстве или на плоскости. Под описанием трёхмерного объекта понимается знание о положении каждой точки объекта в пространстве в любой момент времени. Положение точек в пространстве удобно описывать с помощью ДСК.

Используются три направленные прямые линии (оси), не лежащие в одной плоскости. Они пересекаются в одной точке "--- \textbf{начале координат}. Выбирается на осях единица измерения. Тогда положение любой точки в пространстве можно описать через координаты этой точки. Для практических расчётов удобнее всего испоьзовать ортогональную СК.
\begin{equation}\label{eq:system}
    \left\{
    \begin{aligned}
    (x - x_{1})(y_{2} - y_{1}) &= (x_{2} - x_{1})(y - y_{1}) \\
    (y - y_{1})(z_{2} - z_{1}) &= (y_{2} - y_{1})(z - z_{1}) \\
    (z - z_{1})(x_{2} - x_{1}) &= (z_{2} - z_{1})(x - x_{1})
    \end{aligned}
    \right.
\end{equation}
Система \ref{eq:system} определяет прямую в трёхмерном пространстве.

Для получения динамических изображений необходимо применять преобразование координат точек объекта. Такими преобразованиями являются переносы, повороты, масштабирование, зеркальное отражение. Новые координаты получаются путем преобразования старого базиса координат в новый базис. Как правило, данные преобразования являются линейными. 
\begin{equation}
    m_{i} = f_{i}(k_{1}, k_{2}, \ldots, k_{n})
\end{equation}
где $m_{i}$ – координаты нового базиса, $k_{i}$ – координаты старого базиса, $f_{i}$ – функция пересчёта $i$-ой координаты. Преобразования также делятся на линейные и нелинейные. Если для всех $i = 1, 2, \ldots, N$ функции $f_{i}$ "--- линейные относительно аргументов $(k_{1}, k_{2}, \ldots, k_{n})$, то $f_{i} = a_{i1}k_{i} + a_{i2}k_{2} +, \ldots, a_{in}k_{n} + a_{in-1}$, где $a_{ij}$ "--- константы, то такие преобразования называют линейными. 

Основные операции (перенос, масштабирование, поворот) можно выполнять с помощью матриц.
\begin{enumerate}
    \item \textbf{Перенос} "--- смещение объекта на заданный вектор.
    \begin{equation}
        T(t_{x}, t_{y}) = 
        \begin{bmatrix}
            1 & 0 & t_{x} \\
            0 & 1 & t_{y} \\
            0 & 0 & 1
        \end{bmatrix}
        \text{"--- в 2D}
    \end{equation}
    \begin{equation}
        T(t_{x}, t_{y}, t_{z}) = 
        \begin{bmatrix}
            1 & 0 & 0 & t_{x} \\
            0 & 1 & 0 & t_{y} \\
            0 & 0 & 1 & t_{z} \\
            0 & 0 & 0 & 1
        \end{bmatrix}
        \text{"--- в 3D}
    \end{equation}
    Точка $A(x, y)$ после переноса будет иметь координаты $(x + t_{x}, y + t_{y})$, аналогично для $A(x, y, z)$. \\
    \item \textbf{Масштабирование} "--- изменение размера объекта относительно начала координат.
    \begin{equation}
        S(s_{x}, s_{y}) = 
        \begin{bmatrix}
            s_{x} & 0 & 0 \\
            0 & s_{y} & 0 \\
            0 & 0 & 1
        \end{bmatrix}
        \text{"--- в 2D}
    \end{equation}
    \begin{equation}
        S(s_{x}, s_{y}, s_{z}) = 
        \begin{bmatrix}
            s_{x} & 0 & 0 & 0 \\
            0 & s_{y} & 0 & 0 \\
            0 & 0 & s_{z} & 0 \\
            0 & 0 & 0 & 1
        \end{bmatrix}
        \text{"--- в 3D}
    \end{equation}
    Точка $A(x, y)$ после масштабирования будет иметь координаты $(s_{x} \cdot x, s_{y} \cdot y)$, аналогично для $A(x, y, z)$. Если $s_{x} = s_{y} = s_{z}$, преобразование сохраняет пропорции. Отрицательные значения $s_{x}, s_{y}, s_{z}$ дают отражение объекта.
    \item \textbf{Поворот} "--- изменение ориентации объекта вокруг заданной оси.
    \begin{equation}
        R(\theta) = 
        \begin{bmatrix}
            \cos\theta & -\sin\theta & 0 \\
            \sin\theta & \cos\theta & 0 \\
            0 & 0 & 1
        \end{bmatrix}
        \text{"--- в 2D}
    \end{equation}
    \begin{equation}
        R_{x}(\theta) = 
        \begin{bmatrix}
            1 & 0 & 0 & 0 \\
            0 & \cos\theta & -\sin\theta & 0 \\
            0 & \sin\theta & \cos\theta & 0 \\
            0 & 0 & 0 & 1
        \end{bmatrix}
        \text{"--- в 3D вокруг X}
    \end{equation}
    \begin{equation}
        R_{y}(\theta) = 
        \begin{bmatrix}
            \cos\theta & 0 & \sin\theta & 0 \\
            0 & 1 & 0 & 0 \\
            -\sin\theta & 0 & \cos\theta & 0 \\
            0 & 0 & 0 & 1
        \end{bmatrix}
        \text{"--- в 3D вокруг Y}
    \end{equation}
    \begin{equation}
        R_{z}(\theta) = 
        \begin{bmatrix}
            \cos\theta & -\sin\theta & 0 & 0 \\
            \sin\theta & \cos\theta & 0 & 0 \\
            0 & 0 & 1 & 0 \\
            0 & 0 & 0 & 1
        \end{bmatrix}
        \text{"--- в 3D вокруг Z}
    \end{equation}
    Угол задаётся в радианах, важна последовательность поворотов, то есть $R_{x} \cdot R_{y} \neq R_{y} \cdot R_{x}$. \\
\end{enumerate}

Теперь рассмотрим цветовые модели в компьютерной графике. \textbf{RGB} "--- red, green, blue. Модель, преднозначенная для компьютера (экрана компьютера). Три основных (первичных) цвета, из которых создаются все остальные цвета. Диапазон значений: 0--255 (8 бит на канал). Например, код белого цвета имеет вид (255, 255, 255).

\textbf{HSL} "--- Hue, Saturation, Lightness. Модель, где цвет описывается тремя параметрами \textbf{Hue} "--- цветовой оттенок (тон), \textbf{Saturation} "--- интенсивность цвета (насыщенность), \textbf{Lightness} "--- яркость цвета (светлота). \textbf{HSV} "--- модель, где цвет определяется практически теми же параметрами. Hue и Saturation "--- аналогичны, \textbf{Value} "--- яркость самого светлого компонента. Их отличие в том, что модель HSL используют для создания гармоничных цветовых палитр в веб"=дизайне. HSV "--- для работы с яркостью и насыщенностью в графических редакторах\cite{vasilev2005cg}.

\textbf{Глубина цвета} "--- это количество бит, выделяемых для кодирования цвета одного пикселя. Чем выше глубина цвета, тем больше оттенков можно отобразить. Основные из них:
\begin{enumerate}
    \item 1 бит на канал (монохром): 2 цвета. Примером являются старые дисплеи, штрих"=коды.
    \item 8 бит на канал (24 бита в RGB): 256 оттенков на канал, следовательно, 16,7 млн. цветов. Такая глубина цвета является стандартом для большинства дисплеев и форматов (JPEG, PNG).
    \item 16 бит на канал (48 бит RGB): 65,536 оттенков на канал, следовательно, 281 триллион цветов. Используется в профессиональной фотографии.
    \item 24 бита "--- это фото, видео.
    \item 30--48 бит "--- кино, профессиональная графика.
\end{enumerate}