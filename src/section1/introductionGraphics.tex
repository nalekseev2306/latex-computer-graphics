\textbf{Компьютерная графика} "--- это область науки и технологии, связанная с созданием, обработкой и отображением изображений с помощью компьютера. Цель компьютерной графики "--- создание и обработка изображений с помощью компьютеров и программного обеспечения для решения различных задач в разных областях.

Основные задачи:
\begin{enumerate}
    \item Создание и редактирование изображений. Разработка инструментов и методов для изменения графических изображений.  
    \item Визуализация данных. Преобразование данных в визуальные формы для более лёгкого понимания и анализа.  
    \item Анимация и моделирование. Создание движущихся изображений и трёхмерных моделей для различных приложений.  
    \item Интерактивность. Разработка графических интерфейсов, которые позволяют пользователям взаимодействовать с изображениями и данными.  
    \item Фотореалистичность. Достижение максимальной реалистичности в изображениях, чтобы они выглядели как реальные фотографии или натурные сцены.
\end{enumerate}

Развитие графики прошло через несколько важных этапов, отмеченных существенными технологическими изменениями и инновациями.

\textbf{Зарождение} (1950"=е -- 1960"=е гг.): представляет собой захватывающий этап в истории информационных технологий, когда были сделаны первые шаги в направлении создания и визуализации графических изображений при помощи вычислительной техники.

На ранних этапах развития, когда компьютеры только начали появляться, графические программы разрабатывались для решения простых задач. В 1950-х годах, в рамках проекта \flqq SAGE\frqq\ (Semi"=Automatic Ground Environment), использовались первые визуализации для отслеживания объектов на экранах мониторов.

Одним из важных событий стала разработка Графического дисплейного терминала (Graphics Display Console) в 1960 году исследовательской группой Массачусетского технологического института (MIT). Этот инженерный терминал позволял пользователям взаимодействовать с вычислительной машиной и рисовать изображения.

В это время были разработаны первые алгоритмы для рисования линий и кривых, такие как алгоритм Брезенхэма, который оставался важным на протяжении многих лет. Первые графические стандарты, такие как FORTRAN Graphics Systems (FGR) и CORE (Computer Output to Real"=Time Equipment), появились для стандартизации работы с графикой.

\textbf{Эра растровых изображений} (1970"=е гг.): ознаменовалась важными достижениями, которые положили начало современному развитию машинной визуализации. В этот период активно исследовались и внедрялись новые методы обработки и отображения растровых изображений.

Одним из ключевых событий было создание аппаратно-программного комплекса Xerox Alto в 1973 году. Этот компьютер включал в себя монитор с разрешением 606 на 808 пикселей и предоставлял новые возможности для работы с растровыми элементами. Аппаратно-программный комплекс стал важным шагом в развитии графических интерфейсов.

Айвен Сазерленд представил проект \flqq Sketchpad\frqq\, первую систему интерактивной графики. Она позволяла пользователям создавать изображения при помощи светового пера и стала прорывом в направлении цифрового моделирования и рисования.

\textbf{Возникновение трехмерной графики} (1980"=е -- 1990"=е гг.): период интенсивного развития визуальных технологий, когда компьютеры стали способными создавать и отображать трехмерные сцены и объекты. Это также этап, когда инженеры активно развивали первые графические редакторы. В 1985 году была представлена программа Paint, вместе с операционной системой Windows. Важным было постепенное внедрение цветной растровой графики "--- это позволило создавать более выразительные и креативные визуальные композиции.

Одним из первых значимых событий стало создание цифровой графики для фильма \flqq Терминатор 2: Судный день\frqq\ в 1991 году, где впервые успешно использовались компьютерные эффекты для создания трехмерных персонажей и сцен.
Появился стандартный 3D"=графический интерфейс GKS (Graphics Kernel System), предоставляющий программистам стандартизированный доступ к функциям трехмерной графики.

В это время также начали разрабатываться первые программные интерфейсы для трехмерной графики, такие как OpenGL и DirectX. Эти библиотеки стали основой для разработки приложений и игр.

С появлением первых графических рабочих станций, таких как Silicon Graphics (SGI), графика стала доступной для широкого круга профессионалов. SGI привнесла мощные графические возможности, которые нашли применение в кинопроизводстве, медицинской визуализации и других областях.
Трехмерная графика стала основой для создания игр и развития виртуальной реальности. В 1990"=е годы появились первые успешные 3D"=игры, такие как Doom и Quake, которые стали миллионными бестселлерами и задали стандарты качества в игровой индустрии.

\textbf{Графика в современных технологиях} (2000"=е гг. -- настоящее время): внушительное развитие, затронувшее разнообразные сферы человеческой деятельности. В игровой индустрии наблюдается стремительный прогресс в направлении гиперреализма, с улучшением текстур, освещения и визуальных эффектов. Технологии трассировки лучей в реальном времени, применяемые в современных играх, обеспечивают более точное и реалистичное отображение сцен.

В киноиндустрии визуальные эффекты нового поколения, созданные с использованием компьютерных технологий, позволяют реализовать самые смелые кинематографические идеи. Моушн"=дизайн и цифровые станции приобретают все большее значение для создания анимированных персонажей и сцен.

В области виртуальной и дополненной реальности графика играет ключевую роль в создании иммерсивных визуальных сцен и виртуальных миров. Это применение также расширяется на обучение, медицину и визуализацию данных.

Сфера образования и исследований использует графику для создания обучающих симуляций и визуализаций научных данных. В веб"=дизайне и мобильных приложениях акцент сделан на анимации, интерактивности и адаптивности графики под различные устройства.

Вся окружающая нас графическая информация сделана при помощи компьютера: книги, журналы, упаковки, обои, плакаты, инструкции, сайты и приложения и т. д. Часто ручная графика гармонично интегрируется в компьютерную: иллюстратор рисует изображение тушью, акварелью или любым другим инструментом, а затем оцифровывает его, встраивает в макет и обрабатывает. Такая интеграция превращает ручной рисунок в компьютерную графику.

Рассмотрим области применения компьютерной графики:
\begin{enumerate}
    \item Дизайн "--- баннеры, обои для экранов, лендинги, сайты и т.д.
    \item Анимация и игровая индустрия "--- ролики, фантазийные миры и персонажи.
    \item Полиграфия и реклама "--- от буклетов до огромных 3D"=проекций. 
    \item Киноиндустрия "--- спецэффекты.
    \item Промышленность "--- 3D"=моделирование.  
    \item Архитектура "--- визуализация проектов и создание рендеров.
    \item Живопись "--- цифровые картины.
    \item Медицина "--- обучающие симуляторы, например стоматологический \\ VirTeaSy для создания имплантов и протезов с помощью 3D"=технологий.   
    \item Образование и культура "--- симуляторы и проекты дополненной реальности, например Google Arts \& Culture.
\end{enumerate}