\textbf{FPS} (Frames Per Second) "--- это количество кадров в секунду, которое отображается на экране во время воспроизведения видео или игры. Показатель FPS отражает, насколько плавно воспринимается изображение глазом пользователя. Чем выше значение FPS, тем более плавным и комфортным выглядит движение на экране.

Особенно важную роль FPS играет в играх. Так, FPS в играх "--- это показатель, который указывает, сколько кадров рендерится и отображается на экране за одну секунду. Благодаря высокому показателю FPS, картинка в играх является более плавной и приятной глазу. Однако, всё не так линейно, и здесь есть нюансы. Так, важен не только сам показатель в игре, но и его стабильное поддерживание. На устройстве или в игре в процессе может возникнуть проблема, из"=за которой FPS может резко снизиться. Если ошибки происходят периодически, то это гораздо хуже сказывается на качестве игры, чем низкий, но стабильный FPS.

Немаловажную роль в поддержке FPS играют показатели самой техники, то есть мониторов, с которых происходит воспроизведение. Частота обновления экрана оказывает сильное влияние на то, как быстро будет обновляться картинка. Например, частота экрана в 60 Гц способна менять кадр каждый 16 миллисекунд, а в 144 Гц "--- каждые 6 миллисекунд.

Разумеется, такое понятие, как FPS применяется и в съемке кино. Самая распространённая частота кадров в фильмах "--- это 24 FPS. Именно данный показатель является самым комфортным для зрителя при просмотре фильма. Показатель FPS в 24 единицы поддерживает реалистичность и плавность повествования, так как именно с такой частотой видят наши глаза в повседневной жизни. Усиленная плавность может казаться \flqq фальшивой\frqq\, так как в такой ситуации в сцене мы замечаем гораздо больше элементов, чем могли бы сделать это в реальности. Также у такой низкой частоты есть и практическая цель: съемка и монтаж видео с более высокой частотой кадров требуют больше места и производительности.

Если в случае с играми, плавность играет очень важную роль для поддержания качества игры и мгновенной отдачи действия игрока, то главная задача FPS в фильмах "--- это поддержание реалистичности кадра и сохранения художественного стиля.

Что такое вертикальная синхронизация и как она помогает с оптимизацией FPS? \textbf{Вертикальная синхронизация}, или \textbf{V"=Sync} "--- это технология, предназначенная для синхронизации частоты обновления монитора с частотой кадров, которую генерирует видеокарта. Основная цель V"=Sync "--- устранить разрывы изображения, которые могут возникать, когда видеокарта выводит кадры быстрее, чем монитор может их отображать.

Когда видеокарта генерирует кадры быстрее, чем монитор может их отображать, происходит разрыв изображения. Это означает, что на экране одновременно отображаются части нескольких кадров, что приводит к визуальным артефактам. V"=Sync решает эту проблему, заставляя видеокарту ждать, пока монитор не будет готов к отображению следующего кадра. Это достигается путем синхронизации частоты обновления монитора с частотой кадров видеокарты. Например, если монитор обновляется с частотой 60 Гц, V"=Sync заставляет видеокарту генерировать кадры с частотой не более 60 FPS, что предотвращает разрывы изображения.

Преимущества вертикальной синхронизации:
\begin{itemize}
    \item Устранение разрывов изображения. Основное преимущество "--- это устранение разрывов изображения, что делает игровой процесс более плавным и приятным для глаз.
    \item Стабильность изображения. Помогает поддерживать стабильную частоту кадров, что может быть полезно в играх, где важна плавность и стабильность.
\end{itemize}

Недостатки:
\begin{itemize}
    \item Задержка ввода (input lag). Одним из основных недостатков является увеличение задержки ввода. Это может быть критично в играх, где важна быстрая реакция, таких как шутеры или файтинги. Задержка ввода может привести к тому, что действия игрока будут отображаться на экране с небольшим запозданием, что может повлиять на игровой процесс.
    \item Падение производительности. В некоторых случаях может приводить к падению производительности, особенно если видеокарта не может поддерживать стабильную частоту кадров, равную частоте обновления монитора. В таких ситуациях частота кадров может резко падать.
\end{itemize}