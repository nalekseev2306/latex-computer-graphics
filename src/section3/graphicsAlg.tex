Алгоритм Брезенхэма для рисования линии "--- это алгоритм, определяющий, какие точки n"=мерного растра нужно закрасить, чтобы получить близкое приближение прямой линии между двумя заданными точками . Реализация алгоритма на C++:
\begin{lstlisting}[style=myStyle]
void drawLine(int x1, int y1, int x2, int y2) {
    const int deltaX = abs(x2 - x1);
    const int deltaY = abs(y2 - y1);
    const int signX = x1 < x2 ? 1 : -1;
    const int signY = y1 < y2 ? 1 : -1;
    int error = deltaX - deltaY;
    setPixel(x2, y2);
    while (x1 != x2 || y1 != y2) {
        setPixel(x1, y1);
        int error2 = error * 2;
        if (error2 > -deltaY) {
            error -= deltaY;
            x1 -= signX;
        }
        if (error2 < deltaX) {
            error += deltaX;
            y1 -= signY;
        }
    }
}
\end{lstlisting}

Также существует алгоритм Брезенхэма для окружностей. По методу построения он похож на рисование линии. Строится дуга окружности для первого квадранта, а координаты точек окружности для остальных квадрантов получаются симметрично. На каждом шаге алгоритма рассматриваются три пикселя, из них выбирается наиболее подходящий путём сравнения расстояний от центра до выбранного пикселя с радиусом окружности. Реализация на C++: \cite{bresenham}
\begin{lstlisting}[style=myStyle]
void drawCircle(int x0, int y0, int radius) {
    int x = 0;
    int y = radius;
    int delta = 1 - 2 * radius;
    int error = 0;
    while (y >= 0) {
        setPixel(x0 + x, y0 + y);
        setPixel(x0 + x, y0 - y);
        setPixel(x0 - x, y0 + y);
        setPixel(x0 - x, y0 - y);
        error = 2 * (delta + y) - 1;
        if (delta < 0 && error <= 0) {
            ++x;
            delta += 2 * x + 1;
            continue;
        }
        if (delta >= 0 && error > 0) {
            --y;
            delta += 1 - 2 * y;
            continue;
        }
        ++x;
        delta += 2 * (x - y);
        --y;
    }
}
\end{lstlisting}