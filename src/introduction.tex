Представление данных на мониторе компьютера в графическом виде впервые было реализовано в середине 50"=х годов для больших ЭВМ, применявшихся в научных и военных исследованиях. С тех пор графический способ отображения данных стал неотъемлемой принадлежностью большинства компьютерных систем, в особенности персональных. Графический интерфейс пользователя сегодня является стандартом \flqq де"=факто\frqq\ для программного обеспечения разных классов, начиная с операционных систем.

Существует специальная область информатики, изучающая методы и средства создания и обработки изображений с помощью программно"=аппаратных вычислительных комплексов "--- компьютерная графика. Она охватывает все виды и формы представления изображений, доступных для восприятия человеком либо на экране монитора, либо в виде копии на внешнем носителе (бумага, кинопленка, ткань и прочее). Без компьютерной графики невозможно представить себе не только компьютерный, но и обычный, вполне материальный мир. Визуализация данных находит применение в самых разных сферах человеческой деятельности. Для примера назовем бизнес и экономику, медицину (компьютерная томография), научные исследования (визуализация строения вещества, векторных полей и других данных), моделирование тканей и одежды, опытно"=конструкторские разработки.

Хотя компьютерная графика служит всего лишь инструментом, ее структура и методы основаны на передовых достижениях фундаментальных и прикладных наук: математики, физики, химии, биологии, статистики, программирования и множества других. Это замечание справедливо как для программных, так и для аппаратных средств создания и обработки изображений на компьютере. Поэтому компьютерная графика является одной из наиболее бурно развивающихся отраслей информатики и во многих случаях выступает \flqq локомотивом\frqq\, тянущим за собой всю компьютерную индустрию.