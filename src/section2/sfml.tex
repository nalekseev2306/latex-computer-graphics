Познакомимся с библиотекой SFML на простых практических примерах. Для этого нужно скачать бибилотеку на компьютер с официального сайта: \url{https://www.sfml-dev.org/download/}, затем понадобиться провести подключение всех компонентов библиотеки в рабочую область.

Рассмотрим простейшие функции, которые помогут написать первую тестовую программу (вывод простых фигур на экран). Для начала понадобится подключить заголовочный файл, позволяющий работать непосредственно с графикой:

\begin{lstlisting}[style=myStyle, numbers=none]
#include <SFML/Graphics.hpp>
\end{lstlisting}

После подключения появляется возможность использовать новые функции. В \texttt{main} инициализируем новое окно размером 800 на 600 пикселей следующим образом:

\begin{lstlisting}[style=myStyle, numbers=none]
sf::RenderWindow window(sf::VideoMode(800, 600), "Shapes");
\end{lstlisting}

Теперь определимся с формой, размером, цветом и позицией отображаемых фигур и инициализируем их, задав нужные параметры с помощью простых и интуитивно понятных методов:

\begin{lstlisting}[style=myStyle]
sf::CircleShape circle(50.f);
circle.setFillColor(sf::Color::Red);
circle.setPosition(100.f, 100.f);

sf::RectangleShape rect(sf::Vector2f(120.f, 60.f));
rect.setFillColor(sf::Color::Blue);
rect.setPosition(300.f, 200.f);
\end{lstlisting}

После инициализации всех необходимых элементов можно приступить к созданию \textbf{основного цикла}, который реализует всю логику программы и обеспечивает её работу. Он включает в себя такие этапы как: обработка событий, обновление логики, отрисовка. В нашей программе отсутствует логика, поэтому мы пропускаем второй этап:

\begin{lstlisting}[style=myStyle]
while (window.isOpen()) {
    sf::Event event;
    while (window.pollEvent(event)) {
        if (event.type == sf::Event::Closed)
            window.close();
    }

    window.clear(sf::Color::Black);
    window.draw(circle);
    window.draw(rect);
    window.display();
}
\end{lstlisting}

Обязательным элементом является наличие цикла, обрабатывающего события "--- \texttt{\color{blue}while} \texttt{window.pollEvent(event)}, если бы его не было, то программу было бы невозможно просто так закрыть. Полный код (Листинг \ref{lst:shapes}) и результат программы (Рисунок \ref{fig:shapes}).
