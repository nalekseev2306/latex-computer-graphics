\textbf{Графические библиотеки} "--- это наборы заранее написанного кода, которые упрощают процесс отрисовки графики в приложениях. Они предоставляют набор функций для выполнения распространённых графических задач, что значительно сокращает объём шаблонного кода, который приходится писать разработчикам. Также они позволяют программистам сосредоточиться на разработке своих приложений, а не на низкоуровневых деталях рендеринга графики.

Существует множество разных графических библиотек, каждая из которых имеет свои плюсы и минусы, а также подходит для определённых задач. Поэтому для решения вашей задачи для начала нужно определиться с библиотекой, зная её преимущества и недостатки. Перечислим примеры таких библиотек для C++ и остановимся на некоторых из них, чтобы рассмотреть их подробнее \cite{graphics_libraries}:
\begin{itemize}
    \item SDL;
    \item SFML;
    \item OpenGL;
    \item Allegro;
    \item Juce;
    \item Qt и другие.
\end{itemize}

\textbf{SFML} (Simple and Fast Multimedia Library) "--- это одна из самых быстрых и простых библиотек C++ для реализации 2D"=графики, которая является мультимедийной и позволяет работать с системой, непосредственно с графикой, окнами, аудио и сетью, что помогает упростить разработку игр и мультимедийных приложений. Кроме того она является мультиплатформенной и может быть запущена на распространённых операционных системах: Windows, Linux, macOS, Android и iOS. SFML представленна и на других языках кроме C++: C, .NET, Java, Ruby, Python, Go и других, что позволяет расширить использование данной библиотеки в разных проектах. \cite{sfml_doc}

Стоит выделить главные преимущества SFML:
\begin{itemize}
    \item объектно-ориентированный дизайн, упрощающий использование и понимание;
    \item встроенная поддержка различных графических элементов, таких как фигуры, спрайты и текст;
    \item отлично подходит для начинающих и небольших и средних проектов.
\end{itemize}

\textbf{SDL} (Simple DirectMedia Layer) "--- это кроссплатформенная библиотека для разработки, предназначенная для низкоуровневого доступа к аудио, элементам ввода и графическому оборудованию. SDL поддерживается на всех доступных платформах (Windows, macOS, Linux, iOS и Android), а также возможно использование на разных языках: C\#, Python, Rust и другие. Используется библиотека для разработки игр, видеоплееров и других мультимедийных приложений. \cite{sdl_doc}

Некоторые возможности SDL:
\begin{itemize}
    \item аппаратно-ускоренный вывод 2D- и 3D-графики;
    \item простота интеграции с другими библиотеками, такими как OpenGL, \\ Direct3D или Vulkan;
    \item поддержка шейдеров, текстур и других средств работы с графикой. 
\end{itemize}

\textbf{OpenGL} (Open Graphics Library) "--- это библиотека, которая является одной из самых популярных графических \textbf{прикладных программных интерфейсов} (API – Application Programming Interface) для разработки приложений в области 2D"= и 3D"=графики. Она была разработана ещё в 1992 году на основе библиотеки IRIS GL и до 2017 года активно поддерживалась и развивалась до появления Vulkan API. Однако OpenGL всё ещё широко применяется в различных сферах. Стоит отметить, что библиотека сложнее SFML и SDL, но предлагает широкие возможности и гибкость для продвинутых графических приложений. \cite{opengl_doc}

Рассмотрим особеннсоти OpenGL.
\begin{itemize}
    \item \textit{Стабильность.} Дополнения и изменения реализуются таким образом, чтобы сохранить совместимость с разработанным ранее программным обеспечением.
    \item \textit{Мультиплатформенность} Приложения, использующие OpenGL, гарантируют одинаковый визуальный результат вне зависимости от типа используемой операционной системы и организации отображения информации. Кроме того, эти приложения могут выполняться как на персональных компьютерах, так и на рабочих станциях и суперкомпьютерах.
    \item \textit{Легкость применения.} Стандарт OpenGL имеет продуманную структуру и интуитивно понятный интерфейс, что позволяет с меньшими затратами создавать эффективные приложения, содержащие меньше строк кода, чем с использованием других графических библиотек.
\end{itemize}